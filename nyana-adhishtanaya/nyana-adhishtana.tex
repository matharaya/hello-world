%!TEX TS-program = xelatex
%!TEX encoding = UTF-8 Unicode

\documentclass[a4paper,16pt]{article}

% PACKAGES
\usepackage{tabulary}
\usepackage{polyglossia}
%\usepackage[Ligatures=TeX]{fontspec}
\usepackage{fontspec,xunicode,xltxtra}
\usepackage{caption}
\captionsetup{labelformat=empty}


\defaultfontfeatures{Scale=MatchLowercase,Mapping=tex-text}
%\usepackage{longtable}

% OPEN DOCUMENT

\begin{document}

% SPECIFY UNICODE FONT
%\nopagenumbers
\setmainfont{Arial}
%\setmainfont[Ligatures=TeX]{BhashitaComplex}
% SPECIFY ADDITIONAL FONTS FOR GLYPHS NOT INCLUDED IN UNICODE FONT

%\newfontfamily{\Inuktitut}{Euphemia UCAS}
%\newfontfamily{\Khmer}{Khmer MN}
\newfontfamily{\Sinhala}{BhashitaComplex}
%\newfontfamily{\Sinhala}{IskoolaPota}

\newfontfamily{\Arial}{Arial}
%\voffset -1 in
%\hoffset 2 in
\hsize 7 in
\vsize 8 in
 {\Huge \Sinhalaන අෂ\kern-0.13emඨානය}\\
 {\huge
\begin{quote}
\Arial{``}\kern-0.1em\Sinhala{   5 ළ උදයබ\kern-0.12emබය නය වැට   5  වුට\kern-0.13emඨානය (ඉ නැ ම )  වේවා!}\kern-0.12em\Arial{''}
\end{quote}



{\noindent \Sinhala{මිනිත්තු ගණන අඩුවැඩි කරමින් කිහිප වරක්  අධිෂ\kern-0.13emඨාන කරන්න}\Arial \thinspace --- \thinspace\Sinhala{(මිනිත්තු 3,6,8,10 වනතෙක්\thinspace)} \\


{\noindent \Sinhala{මෙය සාර්ථක නම් මේ අන්දමින්ම අනෙකුත් නවලටද  අධිෂ\kern-0.13emඨාන කරන්න.\underline{සංඛාරුපෙක\kern-0.12emඛා නයද} හරිහැටි වැටහේ නම් එතැන දී මෙසේ ථම මා නයට  අධිෂ\kern-0.13emඨාන කරන්න.}}

\begin{quote}
\Arial{``}\Sinhala{විදර්නා  න පිළිවෙළින් මගේ සිත සෝවාන් මා නයට පැමිණේවා!සෝවාන් ඵලය ත වේවා!}\Arial{''}
\end{quote}

{\noindent \Sinhala{විශේෂ අත්දැකීමක් ලැබුන හොත් එය ඵලය දැයි නිසැක වීමට මෙසේ අධිෂ\kern-0.13emඨාන කර බලන්න.}}
\begin{quote}
\Arial{``}\Sinhala{ මිනිත්තු 5 තුළ මට සෝවාන් ඵලය  ප්‍රත්‍යවී මිනිත්තු {5\,කින්} නැගී සිටීම වේවා!}\Arial{''}
\end{quote}
{\noindent \Arial\Sinhala{(මිනිත්තු ගණන අඩුවැඩි කරමින් නැවත නැවැත   අධිෂ\kern-0.13emඨාන කරන්න)}}}
}
%\newpage
\hsize 9 in
\hoffset -1.3 in
%\voffset -1 in
  \noindent
\begin{table}
\huge
\caption{ \Huge \Sinhala{විදනා න }}
%\begin{center}
  \begin{tabular}{rll}
  &&\\
  1&\Sinhala{උදයබ\kern-0.1emබය ණය }&\Sinhala{ --- සංස්කාරයන්ගේ ඇතිවීම-නැතිවීම දැකීම}\\
  2&\Sinhala{භංග  ණය }&\Sinhala{ --- නැතිවීම බිඳීමක් ලෙස ප්‍රකට වීම}\\
   3&\Sinhala{භයතුපට\kern-0.1emඨාන ණය }&\Sinhala{ --- ඒ තුළින් සංස්කාරයන්ගේ  බියජනක}\\
     &                                                                &\Sinhala{\hspace{.35in} බව වැටහීම }\\
   4&\Sinhala{ආදීනව  ණය }&\Sinhala{ --- ඒවායේ ආදීනව-දොස් ප්‍රකට වීම}\\
     5&\Sinhala{නිබ\kern-0.13emබි ණය }&\Sinhala{ --- ඒ තුළින් සියලු සංස්කාර පිළිබඳ  }\\
     &                                                                &\Sinhala{\hspace{.35in} කලකිරීම }\\
      6&\Sinhala{මුඤ\kern-0.1emචිතුකමතාණය }&\Sinhala{ --- ඒවායෙන් මිදෙනු කැමැත්ත}\\
       7&\Sinhala{පටිසංඛා ණය }&\Sinhala{ --- සංස්කාරයන් අනිත්‍ය,දුකඛ,අනාත\kern-0.13emම}\\
        &                                      & \Sinhala{\hspace{.35in} වශයෙන් සැලකීම}\\
        8&\Sinhala{සංඛාරුපෙක\kern-0.12emඛා ණය }&\Sinhala{ --- ත්‍රිලක්‍ෂණ වශයෙන් දැකීම}\\
         9&\Sinhala{අනුලෝම ණය }&\Sinhala{ --- ඉහත විදනා න අටට එකඟව}\\
           &                                      & \Sinhala{\hspace{.35in} ඒවා ස්ථිර කරමින් පහළවන  නය }\\
          10&\Sinhala{ගොභූ ණය }&\Sinhala{ --- සිත සංස්කාර අරමුණෙන් මිදී නිවන}\\
           &                                       & \Sinhala{\hspace{.35in} අරමුණු කර ගැනීම}\\
           11&\Sinhala{මග\kern-0.13emග ණය }&\Sinhala{ --- සෝවාන් මා  නය }\\
            &                                                                &\Sinhala{\hspace{.35in} (ආනන\kern-0.13emතරික සමාධිය) }\\
            12&\Sinhala{ඵල  ණය }&\Sinhala{ --- සෝවාන් ඵලය}\\
             13&\Sinhala{පච\kern-0.13emචවෙක\kern-0.13emඛණ ණය }&\Sinhala{ --- ප්‍රත්‍යක්‍ෂය පිළිබඳව ඉබේම සිදුවන   }\\
             &                                       & \Sinhala{\hspace{.35in} ප්‍රත්‍යවේ\kern-0.49em◌ාව}\\

  \end{tabular}
 % \end{center}
  \end{table}
  




\end{document} 